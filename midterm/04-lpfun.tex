\chapter{LP Fun}

\section{Solving LP with Scipy}

First, we convert the maximization problem into a minimization one so that Scipy knows how to solve it.
This is simply done by multiplying the objective function by $-1$.

\begin{equation}
    \begin{aligned}
    &\min\quad
        & -2x_1 - x_2 & \\
    &s.t.\quad 
        & -2x_1 -x_2    &\leq\ -1 \\
        && x_1 - x_2     &\leq\ 3 \\
        && 4x_1 + x_2    &\leq\ 17 \\
        && x_2           &\leq\ 5 \\
        && -x_1 + x_2    &\leq\ 4 \\
        && x_1, x_2 &\geq 0
    \end{aligned}
\end{equation}

Throwing this into Scipy and we obtain the following solution. Note that the objective value has been multiplied by $-1$ as well to convert it back to the optimal objective value of a maximization problem.

\begin{equation}
    \begin{aligned}
		\mathbf x^* &= \{x_1, x_2\} = \{3, 5\} \\
		z^* &= 11
    \end{aligned}
\end{equation}

\section{Finding the Dual LP Problem}

Since the given problem is already in symmetric form

\begin{equation}
    \begin{aligned}
	& \max\quad& \mathbf{c^T x} & \\
    & s.t.\quad & \mathbf{Ax} &\leq \mathbf b \\
	&& \mathbf x &\geq 0
    \end{aligned}
\end{equation}

The dual is simply

\begin{equation}
    \begin{aligned}
	& \max\quad& \mathbf{b^T y} & \\
    & s.t.\quad & \mathbf{A^T y} &\geq \mathbf c \\
	&& \mathbf y &\geq 0
    \end{aligned}
\end{equation}

Thus, the dual can be written out as

\begin{equation}
    \begin{aligned}
    & \min\quad& -y_1 +3y_2 +17y_3 +5y_4 +4y_5 & \\
    & s.t.\quad 
        &-2y_1 +y_2 +4y_3 -y_5 &\geq 2 \\
        &&-y_1 -y_2 +y_3 +y_4 +y_5 &\geq 1 \\
        &&y_1, y_2, y_3, y_4, y_5 &\geq 0
    \end{aligned}
\end{equation}

\section{Solving the Dual Problem}

Throwing the dual problem (no conversion since already $\min$ problem), we obtain the following solution.

\begin{equation}
    \begin{aligned}
		\mathbf y^* &= \{y_1, y_2, y_3, y_4, y_5\} = \{0, 0, 0.5, 0.5, 0\} \\
		G^* &= 11
    \end{aligned}
\end{equation}

\section{$z^*$ vs. $G^*$}

As can be seen above from the solutions of the primal and dual problem, their objective values are the same where $z^* = G^* = 11$.
This is due to the fundamental principle of duality--Strong duality which states that if the optimal solution to the primal $\mathbf x^*$ exists, then the optimal solution to the dual $\mathbf y^*$ exists and vice-versa. In both cases, $z^* = \mathbf{c^T x^*} = \mathbf{b^T y^*} = G^*$.
(I learned this from Optimization. Thank you Aj. Tipaluck!)
