\section{Closest Pair of Points in 2D}

\subsection{Expected Number of Reboots}

Assume that the points are randomly shuffled at the start but not after each reboot--if we go up to the $i$-th element and rebooted, the first $i$ elements we draw after the reboot will be the same. This means that if we had to reboot after observing the $k$-th element, we will never reboot at the $k$-th element again. Also assume that the closest pair is unique.

Consider the probability of rebooting after observing the $i$-th point $p_i$. This is the same as the probability of $p_i$ being part of a closest pair with another point in the first $i$ points which is $2/i$ since we have looked at $i$ points and $p_i$ could be at either end of the closest pair. Since the $p_1$ and $p_2$ do not cause a reboot since they give the starting $r$,
\[
  \mathbf{E}[\text{Number of reboots}] = \sum_{i=3}^n \frac{2}{i} = 2H_n - 3
\]
where $H_n$ is the $n$-th harmonic number.

\subsection{Real-World Performance}

The implementations can be found \href{https://github.com/nngerncham/cs315_private/tree/main/final/cp_code02}{here}. It should become public after the due date for the exam. The idea for the implementation of linear-time closest pair is based on \href{https://www.cs.cmu.edu/~15451-s15/LectureNotes/lecture16/closest-pair.pdf}{this lecture note} on closest pairs from CMU. Both implementations also use the square distance of integer points in 2D defined by $dist(p_1 := (x_1, y_1), p_2 := (x_2, y_2)) = (x_1 - x_2)^2 + (y_1 - y_2)^2$.
